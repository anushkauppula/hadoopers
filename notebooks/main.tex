\documentclass[11pt]{article}

\usepackage[utf8]{inputenc}
\usepackage{geometry}
\geometry{margin=1in}
\usepackage{graphicx}
\usepackage{hyperref}
\usepackage{amsmath}
\usepackage{float} % for [H] figures

\title{Milestone 3 Report\\
River--Ocean Plastic Pollution Analytics with HYCOM and Global Weather Data}
\author{Sathyanarayana Mallipudi \and Anushka Uppula \and SiddharthSai Myneni}
\date{\today}

\begin{document}

\maketitle

\section{Implementation Steps}

This section describes the end-to-end implementation for our project
\emph{River--Ocean Plastic Pollution Analytics with HYCOM and Global Weather Data}.
The goal is that any reader can reproduce the full pipeline on a similar
environment (local Python + Jupyter + CSV/NetCDF files) without major issues.

\subsection{Environment Setup}

\begin{enumerate}
  \item \textbf{Base environment:}
        We used a Windows 11 laptop with Python~3.12 and Jupyter Notebook
        (via Anaconda). All code was organized in a
        \texttt{river\_ocean\_plastic\_project/} folder with a Git repository.

  \item \textbf{Install Python packages:}
        The analysis relies on standard scientific Python libraries plus a few
        geospatial and NetCDF tools:
\begin{verbatim}
pip install pandas numpy matplotlib seaborn geopandas xarray netcdf4
\end{verbatim}
        Optionally, \texttt{jupyterlab} can be installed if the user prefers it.

  \item \textbf{Project structure:}
        We created the following folder layout:
        \begin{itemize}
          \item \texttt{data/raw/} -- original CSV and NetCDF files
                (rivers, plastic emissions, HYCOM, weather, risk scenarios).
          \item \texttt{data/processed/} -- merged and cleaned CSVs
                (e.g., \texttt{merged\_rivers.csv}, \texttt{country\_risk.csv}).
          \item \texttt{notebooks/} -- Jupyter notebooks for exploration and testing.
          \item \texttt{src/} -- reusable Python scripts (loading, cleaning, plotting).
          \item \texttt{figures/} -- final PNG figures for Milestone 3
                (one per goal plus an intervention summary).
        \end{itemize}

  \item \textbf{Reproducible imports:}
        Each notebook starts with a common import block:
\begin{verbatim}
import os
import numpy as np
import pandas as pd
import matplotlib.pyplot as plt
import seaborn as sns

sns.set(style="whitegrid")
\end{verbatim}
        This ensures that all plots share the same visual style and
        the code runs consistently across goals.
\end{enumerate}

\subsection{Data Acquisition and Storage}

\begin{enumerate}
  \item \textbf{River plastic datasets:}
        We combined several public datasets based on
        Lebreton/Schmidt river plastic emissions
        and country-level mismanaged plastic waste.
        The main CSVs used were:
        \begin{itemize}
          \item \texttt{global\_rivers\_dataset.csv} -- river geometry, mouth coordinates,
                discharge and estimated plastic leakage.
          \item \texttt{country\_mismanaged\_waste.csv} -- mismanaged plastic waste
                and total plastic generation per country.
        \end{itemize}
        These files were saved under \texttt{data/raw/}.

  \item \textbf{HYCOM ocean current data:}
        We downloaded HYCOM ocean model outputs (current magnitude and direction)
        around major river mouths. These were provided as NetCDF files and stored in
        \texttt{data/raw/hycom/}. We used \texttt{xarray} and \texttt{netCDF4} to
        extract current speed at each river mouth location.

  \item \textbf{Global weather dataset:}
        We used the Kaggle \texttt{GlobalWeatherRepository.csv} dataset, which
        provides gridded rainfall, temperature, humidity and air quality
        (PM2.5) at global coordinates. This file was stored as
        \texttt{data/raw/GlobalWeatherRepository.csv}.

  \item \textbf{Risk scenarios (2015--2060):}
        To analyze long-term plastic risk we created or curated a
        \texttt{plastic\_risk\_2015\_2060.csv} file containing, for each
        river or river basin:
        \begin{itemize}
          \item plastic leakage to rivers for 2015 and 2060
                (\texttt{Plastic\_to\_River\_2015\_tons},
                \texttt{Plastic\_to\_River\_2060\_tons}),
          \item mismanaged waste in 2015 and 2060,
          \item a pre-computed \texttt{Risk\_Score\_Change} per country.
        \end{itemize}

  \item \textbf{Central loading script:}
        A single Python module \texttt{src/data\_loading.py} exposes
        functions such as \texttt{load\_rivers()}, \texttt{load\_weather()},
        \texttt{load\_hycom()}, and \texttt{load\_risk\_scenarios()} which
        read the files into \texttt{pandas} or \texttt{xarray} objects.
\end{enumerate}

\subsection{Preprocessing and Data Quality Checks}

We applied consistent cleaning and quality checks before computing any metrics.

\begin{enumerate}
  \item \textbf{River dataset cleaning:}
        Duplicates were removed based on river name and mouth coordinates.
        Units were normalized (e.g., converting plastic emissions into
        metric tons per year). Obvious outliers were inspected manually.

  \item \textbf{Merging river plastic and country data:}
        Rivers were linked to their primary country using a \texttt{Country}
        column and aggregated to produce \texttt{merged\_rivers\_df}, a
        central DataFrame containing river name, country, plastic emissions,
        and HYCOM current speed variables.

  \item \textbf{Weather aggregation:}
        The weather dataset was aggregated to country level by averaging
        rainfall (\texttt{precip\_mm}), temperature, humidity and
        \texttt{air\_quality\_PM2.5} over all grid cells in each country.
        The result was stored as \texttt{weather\_country\_df}.

  \item \textbf{Risk scenario alignment:}
        Risk scenario records were aligned with river and country names
        used in the other datasets. We computed derived variables such as:
        \begin{itemize}
          \item \texttt{Risk\_Score\_Change} (difference between 2015 and 2060),
          \item country-level means of risk change for comparison.
        \end{itemize}

  \item \textbf{Final preprocessed views:}
        The main tables used across all six goals are:
        \begin{itemize}
          \item \texttt{merged\_rivers\_df} -- river plastic + HYCOM currents,
          \item \texttt{country\_plastic\_df} -- country totals of river plastic,
          \item \texttt{weather\_country\_df} -- rainfall and PM2.5 per country,
          \item \texttt{risk\_df} -- risk scenarios 2015 vs 2060,
          \item \texttt{risk\_scores\_df} -- computed plastic–current risk scores.
        \end{itemize}
\end{enumerate}

\subsection{Goal-wise Implementation}

Below we summarize how each goal is implemented in Python.
Function names here correspond to the functions defined in
our main analysis script.

\subsubsection*{Goal 1: Top Plastic-Polluting Rivers and Countries}

\begin{enumerate}
  \item \textbf{Sorting and ranking:}
        From \texttt{merged\_rivers\_df} we sorted rivers by
        \texttt{plastic\_tons} in descending order and selected
        the top 10 rivers.
  \item \textbf{Country aggregation:}
        Using \texttt{groupby("Country")} we summed plastic emissions
        to obtain country-level totals and ordered countries by
        descending contributions.
  \item \textbf{Map preparation:}
        For rivers with mouth latitude and longitude available,
        we plotted river locations on a simple world scatter map with
        marker size proportional to plastic emissions.
  \item \textbf{Figure export:}
        All subplots were combined into a single \texttt{matplotlib}
        figure and exported as
        \texttt{goal1\_top\_rivers\_analysis.png}
        into the \texttt{figures/} folder.
\end{enumerate}

\subsubsection*{Goal 2: Ocean Currents at River Mouths and Plastic--Current Interaction}

\begin{enumerate}
  \item \textbf{Current speed extraction:}
        We joined river mouth coordinates with HYCOM current data
        to derive a \texttt{current\_speed\_ms} column per river.
  \item \textbf{Current speed distribution:}
        A histogram of current speeds showed how many rivers fall
        into very slow, slow, moderate, or fast current regimes.
  \item \textbf{Plastic vs current scatter:}
        We created a scatter plot of \texttt{plastic\_tons}
        vs \texttt{current\_speed\_ms} and computed the Pearson correlation
        between both variables.
  \item \textbf{Current categories:}
        Rivers were assigned a current category using
        \texttt{pd.cut} with bins
        \((0,0.1], (0.1,0.2], (0.2,0.5], (0.5,2]\)
        to study how average plastic emissions vary across speed classes.
  \item \textbf{Visualization:}
        These analyses were combined into the figure
        \texttt{goal2\_ocean\_currents\_analysis.png}.
\end{enumerate}

\subsubsection*{Goal 3: Country-Level Plastic Contributions and Mismanaged Waste}

\begin{enumerate}
  \item \textbf{Country totals:}
        From \texttt{merged\_rivers\_df} we computed total
        \texttt{plastic\_tons} per country and identified the
        top 15 contributors.
  \item \textbf{Linking mismanaged waste:}
        We merged country totals with the mismanaged waste dataset
        based on country names and used the latest available year.
  \item \textbf{Correlation between mismanaged waste and river plastic:}
        We created a scatter plot of mismanaged waste
        (percentage of global mismanaged plastic) vs river plastic
        emissions (tons), and annotated core countries such as
        the Philippines, India, Malaysia and China.
  \item \textbf{Regional placeholder:}
        Because we did not yet integrate population or regional
        mapping data, we reserved one panel in the figure as
        a placeholder explaining that per-capita or regional
        analysis would require additional datasets.
  \item \textbf{Top 5 countries chart:}
        A simple bar chart summarizes the top 5 countries by river
        plastic emissions.
\end{enumerate}

\subsubsection*{Goal 4: Weather Influence on River Plastic Leakage}

\begin{enumerate}
  \item \textbf{Rainfall distribution:}
        Using \texttt{GlobalWeatherRepository.csv} we plotted a histogram
        of rainfall values (\texttt{precip\_mm}) to understand global
        precipitation patterns and highlighted the mean.
  \item \textbf{Temperature vs humidity:}
        We created a scatter plot of temperature vs humidity,
        colored by rainfall, to show typical atmospheric states.
  \item \textbf{Country-level weather vs river plastic:}
        After aggregating weather by country, we merged it with
        country-level river plastic totals and plotted:
        \begin{itemize}
          \item average rainfall vs river plastic emissions,
          \item average PM2.5 vs river plastic emissions.
        \end{itemize}
        For each relationship we computed the Pearson correlation
        and displayed the value directly on the plot.
  \item \textbf{Figure export:}
        All four panels were saved as
        \texttt{goal4\_weather\_influence.png}.
\end{enumerate}

\subsubsection*{Goal 5: Plastic Risk Scenarios (2015 vs 2060)}

\begin{enumerate}
  \item \textbf{Average plastic and waste (2015 vs 2060):}
        From \texttt{risk\_df} we calculated average
        \texttt{Plastic\_to\_River\_2015\_tons} and
        \texttt{Plastic\_to\_River\_2060\_tons}, as well as
        average mismanaged waste in 2015 and 2060.
        These were plotted as paired bars.
  \item \textbf{Risk score change distribution:}
        The \texttt{Risk\_Score\_Change} column was visualized
        with a histogram to show how many rivers or countries
        are projected to experience large increases or decreases
        in risk.
  \item \textbf{Top 10 countries by increased risk:}
        We computed country-level averages of
        \texttt{Risk\_Score\_Change} and plotted the top 10
        countries in a horizontal bar chart.
  \item \textbf{River-level 2015 vs 2060 plastic:}
        A scatter plot of 2015 vs 2060 plastic leakage per river
        shows whether most rivers are projected to increase or
        decrease in plastic emissions.
  \item \textbf{Figure export:}
        The final four-panel figure was saved as
        \texttt{goal5\_risk\_scenarios.png}.
\end{enumerate}

\subsubsection*{Goal 6: Plastic--Current Risk Score for River Mouths}

\begin{enumerate}
  \item \textbf{Risk score definition:}
        For each river with both plastic and current data, we computed:
        \[
          \text{Risk} = 0.6 \times \text{Plastic\_Norm}
                      + 0.4 \times \text{InverseCurrent\_Norm},
        \]
        where plastic emissions and inverse current speed were both
        min–max normalized to \([0,1]\).
        The final score was scaled to a 0--100 range.
  \item \textbf{Risk categories:}
        We assigned discrete categories
        (Low, Medium, High, Critical) using fixed thresholds on the
        0--100 scale.
  \item \textbf{Score distribution and matrix:}
        A histogram shows the distribution of risk scores.
        A scatter plot of plastic emissions vs current speed,
        colored by risk score, illustrates how high plastic and
        slow currents combine into high risk.
  \item \textbf{Top rivers and countries:}
        We plotted the top 10 rivers by risk score and the top
        countries by average river risk (with total plastic plotted
        on a secondary axis).
  \item \textbf{Figure export:}
        These views were exported as
        \texttt{goal6\_risk\_score\_analysis.png}.
\end{enumerate}

\section{Results and Discussion}

In this section we present the results for each goal. For each goal we
include the final figure generated from the main analysis notebook.

\subsection{Goal 1: Top Plastic-Polluting Rivers and Countries}

Figure 1 shows the ranking of rivers and countries by
plastic emissions, along with a simple map of river mouth locations.

\begin{figure}[H]
    \centering
    \includegraphics[width=\textwidth]{goal1_top_rivers_analysis.png}
    \caption{Goal 1 -- Top plastic-polluting rivers, country contributions,
    and global river mouth locations.}
    \label{fig:goal1}
\end{figure}

The Pasig river appears as the highest emitting river in our dataset,
and the Philippines dominates the country-level totals. The map
highlights that many of the worst rivers are clustered in a few coastal
regions, which supports the idea of geographically targeted interventions.

\subsection{Goal 2: Ocean Currents at River Mouths and Plastic--Current Interaction}

Figure 2 summarizes the distribution of ocean current speeds,
the relationship between current speed and plastic leakage, and how
average plastic emissions vary across speed categories.

\begin{figure}[H]
    \centering
    \includegraphics[width=\textwidth]{goal2_ocean_currents_analysis.png}
    \caption{Goal 2 -- Ocean current speeds at river mouths and their
    relationship with river plastic emissions.}
    \label{fig:goal2}
\end{figure}

Although the sample currently includes only a limited number of rivers
with full HYCOM coverage, the plots suggest that very slow currents are
associated with higher plastic accumulation at the river mouth, while
faster currents tend to disperse plastic more quickly offshore. The
correlation value is displayed in the bottom-right panel for transparency.

\subsection{Goal 3: Country-Level Plastic Contributions and Mismanaged Waste}

Figure 3 combines several views of country-level plastic
contributions and mismanaged waste.

\begin{figure}[H]
    \centering
    \includegraphics[width=\textwidth]{goal3_country_contributions.png}
    \caption{Goal 3 -- Country-level plastic contributions, mismanaged
    waste vs river plastic, and top contributing countries.}
    \label{fig:goal3}
\end{figure}

The top-left chart confirms that a small set of countries account for a
large share of global river plastic emissions. The mismanaged waste vs
river plastic scatter indicates that countries with high mismanaged
waste percentages (e.g., the Philippines, India, Malaysia, China)
tend to have substantial river emissions as well. The bottom-left
panel explicitly notes that per-capita or regional analysis is not
included yet and would require additional datasets, which we plan to
explore in future work.

\subsection{Goal 4: Weather Influence on River Plastic Leakage}

Figure 4 illustrates how rainfall, temperature, humidity
and air pollution relate to river plastic emissions.

\begin{figure}[H]
    \centering
    \includegraphics[width=\textwidth]{goal4_weather_influence.png}
    \caption{Goal 4 -- Weather and air quality influence on river
    plastic leakage.}
    \label{fig:goal4}
\end{figure}

The rainfall histogram reveals that most grid cells experience very low
daily rainfall, with occasional heavy rainfall events. The temperature–humidity
scatter shows typical climate regimes captured by the global weather dataset.
At country level, we observe a modest positive correlation between
average rainfall and river plastic emissions, and a very weak
relationship between PM2.5 and river plastic. These findings suggest
that rainfall intensity may influence plastic leakage pulses, but
policy and waste management still play the dominant role.

\subsection{Goal 5: Plastic Risk Scenarios (2015 vs 2060)}

Figure 5 presents our scenario analysis comparing 2015 and
2060 plastic risk.

\begin{figure}[H]
    \centering
    \includegraphics[width=\textwidth]{goal5_risk_scenarios.png}
    \caption{Goal 5 -- Plastic risk scenarios for 2015 vs 2060 and
    top countries by risk increase.}
    \label{fig:goal5}
\end{figure}

On average, both plastic leakage to rivers and mismanaged waste are
projected to increase by 2060 under the baseline assumptions. The
distribution of \texttt{Risk\_Score\_Change} shows a wide spread: some
rivers are expected to improve slightly while others become much worse.
The top-10 list of countries by increased risk includes
Bangladesh, DR Congo, Vietnam, the Philippines, Nigeria and others.
The bottom-right scatter confirms that many rivers move toward higher
plastic loads over time.

\subsection{Goal 6: Plastic--Current Risk Score for River Mouths}

Figure 6 focuses on the combined plastic--current risk
score, which helps prioritize specific rivers and countries.

\begin{figure}[H]
    \centering
    \includegraphics[width=\textwidth]{goal6_risk_score_analysis.png}
    \caption{Goal 6 -- Plastic--current risk score distribution,
    matrix, and top rivers and countries.}
    \label{fig:goal6}
\end{figure}

The histogram shows two extreme examples in the current subset:
one river with very high risk (high plastic, slow currents)
and another with very low risk (lower plastic, faster currents).
The risk matrix visually reinforces the intuition that high plastic
and low current speed create the most problematic conditions.
Country-level averages indicate that Thailand and the Philippines
emerge as high-priority countries once current patterns are
explicitly considered.

\subsection*{Intervention Trade-off Illustration}

In addition to the six core goals, we created a simple intervention
comparison chart (Figure 7) showing the relative
effectiveness and cost of four intervention types:
river mouth barriers, waste management improvements, policy regulations,
and public awareness campaigns.

\begin{figure}[H]
    \centering
    \includegraphics[width=0.9\textwidth]{summary_intervention_analysis.png}
    \caption{Illustrative comparison of intervention effectiveness and
    relative cost for plastic pollution mitigation.}
    \label{fig:interventions}
\end{figure}

Although these values are illustrative, the chart helps connect
our analytical results to possible real-world actions by comparing
which interventions deliver the most impact per unit cost.

\subsection{Discussion of Big Data Metrics}

We evaluated our project in terms of common Big Data characteristics
(5Vs) and practical system metrics.

\subsubsection*{5Vs for Our Project}

\begin{itemize}
  \item \textbf{Volume:}
        The project combines thousands of rivers, global weather
        grids and multi-decadal risk scenarios. While the current
        CSV and NetCDF files fit on a laptop, the pipeline is designed
        to scale to larger HYCOM domains and additional years.

  \item \textbf{Velocity:}
        Our analysis is batch-oriented: new HYCOM or weather data can
        be ingested periodically and the pipeline re-run. The structure
        (separate loading, cleaning and analysis modules) allows future
        extension to near-real-time updates.

  \item \textbf{Variety:}
        We integrated heterogeneous data types:
        vector rivers, gridded ocean currents, gridded weather,
        and country-level socioeconomic indicators. Handling
        multiple coordinate systems and units was a key challenge.

  \item \textbf{Veracity:}
        Data quality was improved through duplicate removal,
        unit checks, outlier inspection and consistency checks
        between river and country totals. We also explicitly show
        where data coverage is partial (for example, limited HYCOM
        coverage) to avoid over-interpreting results.

  \item \textbf{Value:}
        The final outputs provide high-value insights:
        which rivers and countries matter most, how currents
        influence plastic accumulation, how risk could evolve
        toward 2060, and where interventions may be most impactful.
\end{itemize}

\subsubsection*{Other Metrics}

\begin{itemize}
  \item \textbf{Data quality:}
        Cleaning steps ensured consistent units, reduced duplicates
        and handled missing values gracefully. Country name matching
        and alignment across datasets were crucial to avoid silent errors.

  \item \textbf{Latency and processing time:}
        On our environment, loading and processing all datasets,
        including risk scenarios, completes within a few seconds
        per goal, which is acceptable for iterative research workflows.

  \item \textbf{Resource utilization:}
        A single laptop is sufficient for the current datasets.
        However, the code can be moved into a distributed environment
        (e.g., Spark or Dask) if we later incorporate global multi-year
        NetCDF archives at full resolution.

  \item \textbf{Security:}
        All datasets are public and non-sensitive. The project is
        self-contained in a local folder and can be shared via
        a private Git repository without exposing credentials.

  \item \textbf{Cost:}
        Because we relied on public datasets and local computation,
        the monetary cost is essentially zero. Cloud compute could
        be introduced later for larger-scale experiments.
\end{itemize}

\section{Conclusions}

In this Milestone 3 report we implemented an end-to-end big data pipeline
for river–ocean plastic pollution analytics. Starting from raw
river plastic, HYCOM ocean currents, global weather data and
long-term risk scenarios, we:

\begin{itemize}
  \item identified the top plastic-polluting rivers and countries (Goal 1),
  \item quantified how ocean currents at river mouths interact with plastic
        leakage (Goal 2),
  \item analyzed country-level contributions and linked them with mismanaged
        waste (Goal 3),
  \item explored how rainfall, climate and air quality correlate with river
        plastic emissions (Goal 4),
  \item compared plastic risk scenarios between 2015 and 2060 and identified
        countries with the largest projected risk increase (Goal 5), and
  \item built a combined plastic--current risk score to prioritize rivers
        and countries for intervention (Goal 6).
\end{itemize}

Taken together, these goals show how integrating physical oceanography,
weather, and waste management data can support data-driven decisions on
where to invest in plastic pollution mitigation. The current pipeline
already produces actionable rankings and visualizations, and it can be
extended with per-capita metrics, finer HYCOM coverage, and more detailed
policy scenarios in future work.

% ==============================
% REFERENCES
% ==============================
\section*{References}
\begin{enumerate}
    \item Lebreton, L., van der Zwet, J., Damsteeg, J. W., Slat, B., Andrady, A., \& Reisser, J. (2017). River plastic emissions to the world’s oceans. \textit{Nature Communications}, 8, 15611.
    \item Meijer, L. J., van Emmerik, T., van der Ent, R., Schmidt, C., \& Lebreton, L. (2021). More than 1000 rivers account for 80\% of global riverine plastic emissions into the ocean. \textit{Science Advances}, 7(18), eaaz5803.
    \item Geyer, R., Jambeck, J. R., \& Law, K. L. (2017). Production, use, and fate of all plastics ever made. \textit{Science Advances}, 3(7), e1700782.
    \item Hyun, S., \& Hogan, P. (2008). HYCOM (HYbrid Coordinate Ocean Model): Data User’s Manual. Naval Research Laboratory.
    \item Copernicus Marine Service. (2025). Global ocean currents and sea surface height data. \url{https://marine.copernicus.eu}
    \item World Bank. (2022). Global rainfall and climate data. \url{https://climateknowledgeportal.worldbank.org}
    \item Jambeck, J. R., Geyer, R., Wilcox, C., Siegler, T. R., Perryman, M., Andrady, A., ... \& Law, K. L. (2015). Plastic waste inputs from land into the ocean. \textit{Science}, 347(6223), 768--771.
\end{enumerate}

% ==============================
% TEAM INDIVIDUAL CONTRIBUTIONS
% ==============================
\section*{Team Individual Contributions}

\subsection*{Sathyanarayana Mallipudi}
\begin{itemize}
    \item Led overall project coordination and milestone planning.
    \item Developed the preprocessing pipeline for HYCOM and rainfall datasets.
    \item Implemented Goal 1 (River Mouths and Ocean Currents Integration) and Goal 4 (Seasonal Current Patterns).
    \item Created visualizations for current-river interaction and seasonal variability.
    \item Wrote sections on data acquisition, preprocessing, and Goal 1/4 results.
\end{itemize}

\subsection*{Anushka Uppula}
\begin{itemize}
    \item Managed dataset integration and quality assurance.
    \item Implemented Goal 2 (Rainfall Influence) and Goal 3 (Leakage Index).
    \item Performed statistical analysis and correlation studies.
    \item Developed the leakage index formula and country-level risk ranking.
    \item Contributed to the results and discussion sections for Goals 2 and 3.
\end{itemize}

\subsection*{SiddharthSai Myneni}
\begin{itemize}
    \item Designed and implemented Goal 5 (Risk Scenarios) and Goal 6 (Cold-Start Solutions).
    \item Built the clustering model for cold-start river estimation.
    \item Created comparative visualizations for 2015 vs. 2060 scenarios.
    \item Developed the modular workflow for data-poor region analysis.
    \item Drafted the big data metrics discussion and conclusions.
\end{itemize}

\end{document}
